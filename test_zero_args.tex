\documentclass{article}
\usepackage{commandoptionals}

\begin{document}

\section{Testing Zero Arguments}

% Case 1: 0 Optional, 0 Mandatory explicitly
% Syntax: \newcommandoptionals{\cmd}[Opt][Mand]{Body}
\newcommandoptionals{\mycmdZero}[0][0]{ZeroArgs}

Case 1: \mycmdZero \par
Expected: ZeroArgs

\vspace{1em}

% Case 2: Implicit defaults (should be 0 and 0)
% Syntax: \newcommandoptionals{\cmd}{Body} -> Defaults to 0,0
\newcommandoptionals{\mycmdDefault}{DefaultArgs}

Case 2: \mycmdDefault \par
Expected: DefaultArgs

\vspace{1em}

% Case 3: 2 Optional, 0 Mandatory
\newcommandoptionals{\mycmdOptOnly}[2][0][A][B]{Opt: #1, #2}

Case 3: \mycmdOptOnly \par
Expected: Opt: A, B

\vspace{1em}

% Case 4: 0 Optional, 2 Mandatory
\newcommandoptionals{\mycmdMandOnly}[0][2]{Mand: #1, #2}

Case 4: \mycmdMandOnly{X}{Y} \par
Expected: Mand: X, Y

\end{document}
