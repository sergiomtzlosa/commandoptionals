\documentclass{article}
\usepackage{commandoptionals}

\begin{document}

\section{Testing newcommandoptionals}

% Test 1: 1 Optional, 1 Mandatory
% Syntax: \newcommandoptionals{\name}[N][M][Def1]...{Body}
% Syntax: \newcommandoptionals{\name}[Mand][Defaults]{Body}
\newcommandoptionals{\mycmdA}[1]{DefaultA}{Arg1: #1, Arg2: #2}


Test A1 (Default): \mycmdA{Main} \par
Expected: Arg1: DefaultA, Arg2: Main

Test A2 (Specified): \mycmdA[Custom]{Main} \par
Expected: Arg1: Custom, Arg2: Main

\vspace{1em}

% Test 2: 2 Optionals, 1 Mandatory
% Test 2: 2 Optionals, 1 Mandatory
\newcommandoptionals{\mycmdB}[1]{DefOne, DefTwo}{Opt1: #1, Opt2: #2, Mand: #3}


Test B1 (All Defaults): \mycmdB{MainBody} \par
Expected: Opt1: DefOne, Opt2: DefTwo, Mand: MainBody

Test B2 (One Custom): \mycmdB[Cust1]{MainBody} \par
Expected: Opt1: Cust1, Opt2: DefTwo, Mand: MainBody

Test B3 (Two Custom): \mycmdB[Cust1][Cust2]{MainBody} \par
Expected: Opt1: Cust1, Opt2: Cust2, Mand: MainBody

\vspace{1em}

% Test 3: 0 Optionals, 2 Mandatory
% Test 3: 0 Optionals, 2 Mandatory
\newcommandoptionals{\mycmdC}[2]{}{M1: #1, M2: #2}


Test C1: \mycmdC{Apple}{Banana} \par
Expected: M1: Apple, M2: Banana

\vspace{1em}

% Test 4: Max limit check (Optional+Mandatory <= 8)
% This should compile fine: 4 Opt + 4 Mand = 8
% Test 4: Max limit check (Optional+Mandatory <= 9)
% This should compile fine: 4 Opt + 4 Mand = 8
\newcommandoptionals{\mycmdD}[4]{A, B, C, D}{#1-#2-#3-#4 | #5-#6-#7-#8}


% Test D: \mycmdD{E}{F}{G}{H} \par
% Expected: A-B-C-D | E-F-G-H
\typeout{TestD_Result: \mycmdD{E}{F}{G}{H}}

% Extra check for Test A
\typeout{TestA1_Result: \mycmdA{Main}}
\typeout{TestA2_Result: \mycmdA[Custom]{Main}}


\end{document}
