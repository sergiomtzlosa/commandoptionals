\documentclass{ltxdoc}
\usepackage[T1]{fontenc}
\usepackage{lmodern}
\usepackage{commandoptionals}
\usepackage{listings}

\title{The \texttt{commandoptionals} package}
\author{Sergio Martinez-Losa}
\date{\today}

\begin{document}

\maketitle

\begin{abstract}
The \texttt{commandoptionals} package extends standard LaTeX command definitions. It allows users to define commands with a flexible number of optional and mandatory arguments, checking that the total number does not exceed 9.
\end{abstract}

\section{Introduction}
Standard LaTeX \texttt{\textbackslash newcommand} allows for at most one optional argument. The \texttt{xparse} package provides powerful tools for defining complex command interfaces. This package, built on top of \texttt{xparse}, provides a simplified interface \texttt{\textbackslash newcommandoptionals} to define commands with multiple optional arguments and their default values.

\section{Usage}

\subsection{Loading the Package}
\begin{verbatim}
\usepackage{commandoptionals}
\end{verbatim}

\subsection{Defining Commands}
The main command is \texttt{\textbackslash newcommandoptionals}.

\begin{verbatim}
\newcommandoptionals{\<cmd>}[<num_opt>][<num_mand>][<def_1>]...[<def_n>]{<body>}
\end{verbatim}

\begin{itemize}
    \item \texttt{<cmd>}: The command to be defined.
    \item \texttt{<num\_opt>}: Number of optional arguments (default 0).
    \item \texttt{<num\_mand>}: Number of mandatory arguments (default 0).
    \item \texttt{<def\_i>}: Default value for the $i$-th optional argument. You must provide as many default values as optional arguments specified.
    \item \texttt{<body>}: The definition of the command. Arguments are accessed via \texttt{\#1}, \texttt{\#2}, etc. Optional arguments come first, followed by mandatory arguments.
\end{itemize}

Constraint: $\texttt{<num\_opt>} + \texttt{<num\_mand>} \le 9$.

It is possible to have zero optional arguments ($\texttt{<num\_opt>} = 0$) or zero mandatory arguments ($\texttt{<num\_mand>} = 0$), provided the total sum is at least 0 (an empty command is possible, though perhaps not very useful).

\section{Examples}

\subsection{Two Optional, One Mandatory}

\begin{verbatim}
\newcommandoptionals{\mycmd}[2][1][DefaultA][DefaultB]{%
  Opt1: #1, Opt2: #2, Mand: #3%
}

\mycmd{Hello}          % -> Opt1: DefaultA, Opt2: DefaultB, Mand: Hello
\mycmd[UserA]{Hello}   % -> Opt1: UserA,    Opt2: DefaultB, Mand: Hello
\mycmd[UserA][UserB]{Hello} % -> Opt1: UserA, Opt2: UserB, Mand: Hello
\end{verbatim}

\end{document}
